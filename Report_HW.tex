\documentclass[UTF8]{ctexart}
\usepackage{graphicx}
\usepackage{float}
\usepackage{geometry}
\usepackage{fancyhdr}
\usepackage{lastpage}
\usepackage{abstract}

\geometry{left=2.51cm,right=2.51cm,top=3.18cm,bottom=3.18cm}
\begin{document}

\pagestyle{fancy}
\lhead{\includegraphics[scale=1]{sjtu-logo-red.pdf}}  
\rhead{肱骨中上段骨折术后肘关节矫正器} 
\cfoot{第 \thepage\ 页\ 共 \pageref{LastPage} 页} 

\begin{titlepage}
    \begin{center}
        \includegraphics[width=0.8\textwidth]{sjtu-name-blue.pdf}\\[1cm]
        \textsc{\huge \bfseries 课程项目报告}\\[1.5cm]
        \includegraphics[width=0.3\textwidth]{sjtu-badge-blue.pdf}\\[1cm]    
        \textsc{\huge \bfseries 生物医学工程导论作业三}\\[1.5cm]

        \begin{minipage}{0.75\textwidth}
            \begin{flushleft} 
                \LARGE \bfseries{项目名称:肱骨中上段骨折术后肘关节矫正器}\\
                \LARGE \bfseries{小组成员:裴奕博、许世杰、王文才、李启隆}
            \end{flushleft}
        \end{minipage}

    \end{center}
\end{titlepage}


\section{摘要}
    本课程项目的产品名称为:肱骨中上段骨折术后肘关节矫正器。

    近年来,随着人口老龄化的加剧和意外事故发生率的上升,骨折病人的数量正处在逐年增加之中。对于一个骨折病人来说,骨折部位附近在经历一段时间的外固定(石膏、夹板固定)或者内固定(如手术后钢板、钢钉固定),后,会出现关节僵硬、肌肉萎缩等并发症。因此骨折病人在接受治疗后,需要经过一段漫长的康复时间。而这些骨折绝大多数是位于四肢的骨折,康复起来比较困难。因此我们从这一实际需求出发,针对其中肱骨中上段骨干骨折的病人,设计了康复矫正器,运用电、热、机械等多种物理方式,辅助病人进行康复。在设计过程中,我们充分利用现有的技术手段,并对其加以整合,可行性较强。我们相信,我们的产品在获得临床数据支持后,投入实用较为容易。此外,对于其他部位的矫正器械,我们也提出了相应的推广方法和一些展望,使得我们设计的医疗器械能够适应更多的病人和状况。

\section{项目背景}
    \subsection{骨折病人的康复现状及存在的问题}
    \subsection{项目的目的和意义}
    \subsection{项目需要克服的问题}
    %TODO    
\section{项目方案设计}
    \subsection{整体结构设计}
    \subsection{各部分设计}
        \subsubsection{加热装置设计}
        \subsubsection{机械结构设计}
        \subsubsection{电刺激结构设计}
    %TODO
\section{项目难点与解决方法}

\section{项目不足与展望}
    \subsection{项目不足}
    \subsection{项目展望}
    %TODO
    
\end{document}
